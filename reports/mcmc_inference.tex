\documentclass[11pt, a4paper]{article}

% Required packages
\usepackage[utf8]{inputenc}
\usepackage{geometry}
\usepackage{amsmath}
\usepackage{amssymb}
\usepackage{graphicx}
\usepackage{hyperref}
\usepackage{bm} 
\usepackage{microtype}

\newcommand{\keff}{$k_\text{eff}$}
\newcommand{\nuclide}[2]{$^{#1}$#2}

% Set margins
\geometry{left=25mm, right=25mm, top=25mm, bottom=25mm}

% Title and Author
\title{Bayesian Optimisation using microscopic and integral measurements to infer nuclear data parameters}
\author{Daan Houben, Mathieu Hursin}
\date{\today}

\begin{document}

\maketitle

\begin{abstract}
    test 123
\end{abstract}

\section{Introduction}
Nuclear data is considered as the major source of uncertainty in some reactor observables, most notably effective multiplication factor (\keff). The nuclear data available in evaluated nuclear data libraries, such as JEFF-4.0, are a result of a complex fitting procedure including theoretical models, microscopic experiments, expert judgement and sometimes integral experiments. Although integral experiments are often not directly including in the fitting procedure due to the computational cost, they form an important step towards validating the nuclear data vector. In this work, a Bayesian Optimization framework is proposed which enables consolidating microscopic energy dependent measurements with integral experiments to estimate nuclear data parameters. 

Due to the computational cost of evaluating the performance of nuclear data in high-fidelity simulation software, most notably in integral experiments, Gaussian Processes (GPs) are trained. Subsequently, the GPs are used in a Markov Chain Monte Carlo (MCMC) algorithm in order to obtain a posterior estimate of the parameter(s) to infer. Two case studies are presented, first on the $\Gamma_\gamma$ width of \nuclide{53}{Cr} in the keV energy range using data recently published by P. Maroto et al. (2025) \cite{perez-maroto_50cr_2025}, complemented with integral experiments sensitive to \nuclide{53}{Cr}. Secondly, on a hypothetical problem in which the integral experiments are more sensitive to the input parameters.

\section{Background and mathematical motivation}
\subsection{Bayes theorem}

\subsection{Surrogate modelling}

\subsection{Markov Chain Monte Carlo (MCMC)}

\subsection{Likelihood Formulation}

\section{Results}
\subsection{Chromium-53}

\subsection{Uranium-238}

\section{Discussion}

\section{Conclusions and future work}

\bibliographystyle{ieeetr}
\bibliography{references}

\end{document}